\section{Introduction}\label{sec:introduction}
  
  In 1997, Peter Shor developed what have come to be known as “Shor’s Algorithms”, algorithms for solving the prime factorization and discrete log problems.
  \begin{definition}[Discrete log problem]
    \label{def:discretelog}
    Given an abelian group $G$ with a generator $g$, define a function $f(k): \Z \to G$ given by $f(k) \coloneqq g^k$,
    the discrete log problem asks for an efficient algorithm that, given $f(k)$, can find $k$.
  \end{definition}
  \begin{definition}[Prime factorization problem]
    Given a positive integer $n$, the prime factorization problem asks for an efficient algorithm to find all prime numbers $p$ such that $p | n$.
  \end{definition}
  While seemingly innocuous, a vast amount of cryptography relies on the assumption that these problems are hard to compute; i.e., that no efficient algorithm can solve them.
  When Shor's paper contradicted this assumption, it set off a frenzy to repair the very foundations of the cryptography we all rely on.
  
  Since then, most discussion of quantum computers has been mired in fear.
  Is any of our data safe?
  What if someone is recording our communications to decrypt at a later date, once quantum computers become feasible?
  Is it even possible to have secure cryptography in the face of quantum computers?
  These are important questions, and their answers are unknown (though there do exist good candidates for post-quantum cryptography).
  But what about their positive applications?
  Are quantum computers nothing more than a specter haunting cryptography?
  Or can they actually be used for something positive?
%  Yet cryptographers have managed to develop new encryption schemes not relying on discrete log.
%  And fueled by the specter of the quantum computer, companies and governments have slowly but surely put them into practice.
%  Even so, discourse around quantum computers remains altogether negative.
%  Quantum computers are dangerous because they can break cryptography.
%  But can they actually do anything positive?
  
  Grover's algorithm provides a hint to an answer.
  It's a very simple algorithm for a very simple problem: finding a single element in an unsorted set.
  It's a problem that is not even interesting on classical computers, because the optimal solution is so obvious.
  (We will see that later when we formalize the problem.)
  And yet Grover's algorithm manages to be somehow even faster, returning a result in less time than it takes to \emph{even read the input}.
  
  To me, that is incredibly surprising and interesting, as it flies very much in the face of our classical intuition of what a computer can do.
  Shor's algorithm is a more technical result without as much intuitive might: if you told the average person that you had an efficient algorithm for factoring numbers, most would be unimpressed.
  Without having studied the problem, efficient integer factorization doesn't seem like it should be impossibly hard.
  But an algorithm that can find an element in a set without even seeing the whole set?
  I think most would find that truly surprising.