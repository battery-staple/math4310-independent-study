\begin{section}{Quantum Physics Primer}
  \begin{subsection}{Quantum States}
    \label{subsec:quantum-states}
    In classical physics, there are many quantities that together define the state of a system.
    For instance, a particle may have a position, a velocity, a mass, a charge, and more.
    It takes many of these quantities together to form a complete description of the system and predict how it will evolve over time.
    
    In quantum physics, the picture is much simpler: there is only a single ``quantum state'' that describes everything knowable about a system at a given time.
    
    \begin{definition}[Quantum state]
      A quantum state $\psiket$ is a unit vector in some complex Hilbert space $\H$
      \footnote{Although this is in general a Hilbert space, in this report we will generally only consider finite-dimensional Hilbert spaces, so we can just treat it as a regular inner product space.}.
      It is a complete mathematical representation of the system; every predictable facet of the system can be derived from this state.
    \end{definition}
    Any given quantum system has an associated Hilbert space in which its state vectors live, but different systems may be associated with different underlying Hilbert spaces.
  
    NB: At this point, it should be evident that linear algebra is intrinsic to the study of quantum physics and quantum computing in particular.
    We will continue to see this throughout.
    Essentially everything we care about—quantum states, measurements, quantum gates, etc.—will be phrased in the language of linear algebra.
  \end{subsection}
  
  \begin{subsection}{Observables}
    \label{subsec:observables}
    
    Other quantities, like position and momentum, are no longer fundamental to the system and, in fact, no longer even have physical reality.
    In quantum physics, these quantities are called ``observables''.
    Before being measured, they do not exist.
    
    \begin{definition}[Observable]
      An observable $O$ is a measurable quantity of a quantum system.
    \end{definition}
    For instance, the observables of a quantum particle might include its position, momentum, and charge.
    Observables are a physical concept, not mathematical objects.
    However, we will later see how they are formalized mathematically.
    
    More unsettlingly, even when a system's quantum state is known, it is not generally possible to predict the exact result of a given measurement.
    For instance, consider a particle in free space; i.e., in the absence of any gravitational, electromagnetic, or other fields.
    It is not difficult to compute this particle's precise quantum state.
    Nonetheless, if one measures the particle's position, it is possible for the particle to appear anywhere in the universe at all.
    This is not a question of measurement error or incomplete information.
    There is simply a randomness intrinsic to the measurement that physically cannot be reduced.
    This is very surprising!
    Our classical intuition of the world is that objects in a definite state have definite measurable values.
    For instance, if one places a ball on a table, everyone can agree that the ball is, in fact, on the table.
    If nobody moves it and no forces act on it, then we expect that the ball will still be on the table next week.
    It is not the case that every time we blink, the ball magically appears in a different spot.
    Yet this is what happens in quantum physics!
    
    Knowledge of the quantum state is not entirely useless, however.
    Although it is impossible to say exactly where a particle is before it is measured, the state allows us to at least derive a probability distribution of where we expect the particle to appear.
    One can imagine an experiment in which one does the following: first, they put a particle in a particular quantum state.
    Then, they measure its position.
    Although the result of each individual experiment is unpredictable, if it is repeated, the distribution of results will match the predicted distribution.
    We will discuss exactly how to calculate this distribution later on.
  \end{subsection}
  
  \begin{subsection}{Qubits}
    \label{subsec:qubits-and-state-vectors}
    For our purposes in quantum computing, our fundamental objects are called qubits.
    These qubits can be implemented in a variety of ways; for instance, they may be a photon, an electron, or even the state of an entire superconducting circuit.
    Regardless of their physical instantiation, however, a qubit's state always lives in a two-dimensional Hilbert space.
    \begin{definition}[Qubit]
      A qubit is a quantum system whose state lives in a two-dimensional Hilbert space and which can be manipulated by a quantum computer.
    \end{definition}
    From the perspective of quantum computing, this abstract definition means we can ignore their physical reality and use only their mathematical description.
    
    Typically, we operate in a standard basis containing two vectors, $\colvector{1,0}$ and $\colvector{0,1}$.
    In physics notation, these are $\ket{0}$ and $\ket{1}$, respectively.
    (To avoid confusion, note that $\ket{0}$ does \emph{not} represent the zero-vector!)
    Of course, unlike regular bits, qubits are not restricted to only these two states.
    Instead, they can be in any state with unit norm.
    If we write the state as $\ket{\psi}$, a general qubit state therefore looks like:
    $$
    \ket{\psi} = \alpha\ket{0} + \beta\ket{1}
    $$
    for some $\alpha, \beta \in \C$ with $|\alpha|^2 = |\beta|^2$.
    Any state other than $\ket{0}$ or $\ket{1}$ is called a superposition.
    Sometimes these qubits are described as being ``in both the $\ket{0}$ and $\ket{1}$ state at the same time''.
    Mathematically, however, these superposition states are just vectors like any other.
  \end{subsection}
  
  \begin{subsection}{Measurements and Hermitian Operators}
    \label{subsec:measurements}
    Now that we have discussed a bit of the mathematical basis for quantum states, we can return to the question of measurement.
    One important fact, which holds for reasons outside the scope of this report, is that every observable can be identified with a particular Hermitian operator.
    Given an observable $O$ (i.e. position, momentum, energy, etc.), we use the notation $\Ohat$ to denote the unique Hermitian operator associated with it.
    
    \begin{definition}
      A linear operator $A$ is called Hermitian if it is self-adjoint; i.e. $A = A^*$.
    \end{definition}
    \begin{fact}
      \label{fact:observableimplhermitian}
      For every observable quantity $O$, there exists a unique associated Hermitian operator $\Ohat$.
    \end{fact}
    
    As mentioned above, most states do not have a definite value of any particular observable.
    In general, the result of a measurement is unpredictable: multiple measurements of the same state will give different, random, results.
    However, there are certain special states that \emph{do} have definite values of an observable.
    Specifically, for any observable $O$ (again, think position, momentum, or energy), there exist certain states for which any measurement of $O$ will always give the same result.
    \begin{definition}
      A state $\psiket$ is said to have a ``definite value'' of an observable $O$ if a measurement of $O$ on $\psiket$ always gives the same value.
    \end{definition}
    For example, a single photon always has energy $\hbar\omega$, where $\omega$ is the frequency of the light and $\hbar$ is a constant.
    If one were to measure the energy of a single photon, they would always get the same result.
    We would therefore say that a photon has a definite value of energy, equal to $\hbar\omega$.
    
    It turns out that states like this—with definite values of particular observables—are always eigenstates of the operator associated with the observable.
    Formally:
    \begin{fact}
      If $\ket{\psi} \in \H$ is the state of some quantum system in its Hilbert space $\H$, $O$ is some observable, and $\hat{O} \in \L(\H)$ is the Hermitian operator associated with $O$, then:
      $$\Ohat \psiket = \lambda\psiket \text{ for some }\lambda \in \R \iff \psiket\text{ has a definite value of }O$$
      Furthermore, if this happens, the eigenvalue $\lambda$ will always be the result of measuring $O$ on $\psiket$.
    \end{fact}
    
    Continuing with our photon example above, we know that a photon in state $\psiket$ has energy $\hbar \omega$.
    So if we let $\hat{E}$ be the ``energy operator'' (the operator associated with the observable energy), we must have:
    $$\hat{E}\psiket = \hbar\omega\psiket$$
    
    With this fact, along with the fact that all observable operators are Hermitian, we can make some observations.
    For one, we know that the eigenvalues of Hermitian operators are always real.
    This corresponds to the fact that all physical measurements that can be made are always real-valued.
    This is certainly what we expect.
    It would be a confusing world if particles could have a momentum of $2 + i$, for instance!
    
    At this point, one may wonder: what happens if we make a measurement of $O$ on a state $\psiket$ that is \emph{not} an eigenvector?
    We know that these non-eigenvector states do not have a definite value of $O$.
    This means that the result of the measurement is unpredictable, and can take on a number of possible values.
    But which values, exactly?
    
    The answer is a bit odd.
    Consider an arbitrary state $\psiket$.
    We know that if $\psiket$ is an eigenvector of $\Ohat$, then the result of the measurement will be the corresponding eigenvalue of $\Ohat$.
    Interestingly, something similar is true even if $\psiket$ is not an eigenvector:
    \begin{fact}
      For any state $\psiket$ and observable $O$, a measurement of $O$ on $\psiket$ will always result in some eigenvalue $\lambda$.
      It is not possible to measure any value that is not an eigenvalue.
    \end{fact}
    
    This leads to some unintuitive conclusions.
    For instance, in classical physics, we think of energy as a continuum.
    There is no ``smallest amount'' of energy a particle can have, nor is there a maximum.
    Any amount of energy in the range $[0, \inf)$ is theoretically possible.
    However, in many cases, the quantum energy operator $\hat{E}$ has only countably many eigenvalues.
    When this happens, this means there are certain energy values that are simply impossible to achieve.
    It may be possible to have an energy of $0$, $1$, or $2$ but not $1.5$!
    This is very much not true in classical physics.
    
    With this fact stated, a natural next question arises: when we measure a non-eigenvector state, which eigenvalue results are possible?
    We know there will be some randomness in the value measured.
    But how can we describe the distribution of possible measurements?
    
    It turns out that the answer is very simple.
    Consider an arbitrary operator $O$, and recall that $\Ohat$ must be Hermitian.
    Since one can always find eigenvectors of a Hermitian operator that form an orthonormal basis, this implies that any arbitrary state $\psiket$ can always be written in the form:
    $$\psiket = \sum_{i=1}^n{\alpha_i\ket{\psi_i}}$$
    where $\{\ket{\psi_i}\}_{i=1}^n$ are the eigenvectors of $\Ohat$.
    \begin{fact}\label{fact:eigenvalprob}
      Given a quantum state $\psiket$ written in the above form, let $\lambda_i$ be the eigenvalue corresponding to $\ket{\psi_i}$.
      Then a measurement of $O$ on $\psiket$ will give $\lambda_i$ with probability $|\alpha_i|^2$, where $\alpha_i$ is the coefficient of $\ket{\psi_i}$ in the linear combination forming $\psiket$.
    \end{fact}
    Because we know that $\{\ket{\psi_i}\}_{i=1}^n$ form an orthonormal set, we can conclude the following:
    \begin{corollary}
      Suppose $\psiket$, $\ket{\psi_i}$, and $\lambda_i$ are defined as above.
      Then a measurement of $O$ on $\psiket$ will give $\lambda_i$ with probability $|\braket{\psi_i|\psi}|^2$, where $\braket{\psi_i|\psi}$ is physics notation for the inner product $\iprod{\ket{\psi}}{\ket{\psi_i}}$
    \end{corollary}
    
    There's one more fact we'll need before returning to quantum computing, essentially the converse of $\ref{fact:observableimplhermitian}$:
    \begin{fact}
      For any Hermitian operator $\Ohat$, there exists an observable quantity whose possible values are exactly the eigenvalues of $\Ohat$.
    \end{fact}
    This is important to ensure that qubit states are measurable.
    Consider the operator defined by the matrix
    $$\begin{bmatrix}
        1 & 0 \\
        0 & -1 \\
    \end{bmatrix}$$
    in the $\{\ket{0}, \ket{1}\}$ basis.
    Because this an operator diagonal with respect to an orthonormal basis and with real eigenvalues, it is Hermitian.
    This implies we can always physically measure the state of a qubit by carrying out the measurement associated with this operator.
    Our measurement will give $1$ or $-1$, allowing us to ``read the value'' of the qubit.
  \end{subsection}
  
  \begin{subsection}{Multi-Qubit Systems}
    \label{subsec:multi-qubit-systems}
    In any useful quantum algorithm, we will want to have more than one qubit.
    Conveniently, a collection of qubits can be thought of as a new composite quantum system with its own quantum state space.
    In particular, given a set $\{\H_1, \H_2, \dots, \H_n\}$ of Hilbert spaces each corresponding to a qubit's state, the whole system of all $n$ qubits can be described by a state vector in $\H_1 \otimes \H_2 \otimes \dots \otimes \H_n$ (here $\otimes$ is the tensor product).
    
    We will typically denote the state of a system with $n$ qubits in one of two ways.
    The eigenstates will be denoted as $\ket{x_1 x_2 \dots x_n}$, where each of $x_i \in \{0, 1\}$.
    So for instance, the state with four qubits, each in the $\ket{0}$ state is denoted as $\ket{0000}$.
%
    The non-eigenstates will typically be denoted as a $2^n$-dimensional vector, where the first entry is the coefficient of the $\ket{0\dots0}$ state, the second entry is the coefficient of the $\ket{0\dots1}$ state, etc.
    So for instance, the following state of two qubits:
    \begin{align*}
      \psiket &= \frac{1}{2}(\ket{0} \otimes \ket{0}) + \frac{1}{\sqrt{2}}(\ket{0} \otimes \ket{1}) + \frac{1}{2}(\ket{1} \otimes \ket{1})\\
      &= \frac{1}{2}\ket{00} + \frac{1}{\sqrt{2}}\ket{01} + \frac{1}{2}\ket{11}
    \end{align*}
    would be denoted as:
    $$\psiket = \left\langle \frac{1}{2}, 0, \frac{1}{\sqrt{2}}, \frac{1}{2} \right\rangle$$
    
    Sometimes we may also use a variation of the first notation, $\ket{x}$.
    Here, $x$ is a binary string or an integer (represented in binary).
    This can be interpreted just like the first notation, where each integer represents a particular basis state.
    For instance, if $x = 3$, we would have $\ket{x} = \ket{101} = \frac{1}{\sqrt{2}}\langle1, 0, 1\rangle$
  \end{subsection}
  
  \begin{subsection}{Quantum Gates}
    Finally, before we can see the algorithm, we must define quantum gates.
    
    Classical computers are built from different gates—\textsc{AND}, OR, NOT, etc.—which each operate on the $0$s and $1$s stored in the CPU's registers.
    Quantum computers are conceptually no different.
    
    \begin{definition}[Quantum gate]
      A quantum gate is a unitary operator $U \in \L(\H)$ (not necessarily Hermitian nor related to a measurement!), where $\H$ is the Hilbert space of a system of several qubits.
    \end{definition}
    
    A quantum gate simply gets applied to the state of a system of qubits to transform it into a new state.
    The requirement that it be unitary comes from the fact that quantum states always have unit norm.
    It must be unitary so that all output states remain properly normalized.
  \end{subsection}
\end{section}